\chapter{Result and Discussion} \label{chp:result_and_discussion}

The performance of GBM will be evaluated by means of a case study using the test dataset. The test dataset comprises journey data from the whole year of 2021. The first part will focus on performance evaluation of BBM, where the trained model will be used to predict the SOG. The second part focus on the power estimation method using Holtrop-Mennen method. The output of BBM, which is the ship SOG, will be fed to the WBM to estimate the power. For further clarity regarding the methodology, the following steps are taken which are based on the proposed methodology shown in \Cref{fig:flowchart_BBM} and \Cref{fig:flowchart_WBM}. For generation of the BBM, the steps taken are:

\begin{enumerate}
    \setlength\itemsep{0em}
    \item Dataset is loaded.
    \item Identify and remove any anomalies.
    \item Remove static and unneeded features.
    \item Apply speed threshold of 5 knots.
    \item Highly correlated features are combined/removed based on physical and statistical reasoning.
    \item Impute missing values using {\tt KNNImputer}.
    \item Split the dataset into training and testing.
    \item Train the model using the whole dataset with default hyperparameter.
    \item Evaluate model performance using k-fold cross-validation.
    \item Tune the model until the best model is obtained.
    \item For the case study, the best models will be used to predict the SOG using the test dataset.
\end{enumerate}

Subsequently, for FOC calculation, the following steps are taken:

\begin{enumerate}
    \setlength\itemsep{0em}
    \item The test dataset is split into seasonal data. Summer-Fall season and Winter-Spring season which correspond to data for 6 months respectively.
    \item Impute missing values using {\tt KNNImputer}.
    \item SOG is converted to STW.
    \item Calculate calm water resistance $R_{CALM}$.
    \item Calculate added resistance due to wave $R_{AW}$.
    \item Calculate added resistance due to wind $R_{AA}$.
    \item Calculate total effective power $P_E$ using total resistance $R_{TOTAL}$.
    \item Calculate brake power $P_B$ from total efficiencies.
    \item Plot resulting regression line for Power-Speed curve from all models and actual case. 
    \item Calculate the FOC by considering the engine SFOC and operation time.
    \item Plot resulting regression line for FOC-Speed curve from all models and actual case.
    \item Evaluate the performance of the model generated from the regression lines.
\end{enumerate}

\section{Evaluation of BBM}\label{sec:BBM_tree_evaluate}

\subsection*{Model Training and Selection of Optimal Parameter}\label{sec:hpo_select_train}

As mentioned in \Cref{sec:BBM_modelling}. There are 2871 data points available for training. To help narrow the search range of the hyperparameters for the tree-based model, RMSE plots against different values of hyperparameters will be performed. This method was presented in \Cref{sec:hpo}. The hyperparameter will be iteratively tuned until the best model is obtained. The result of the optimal parameter is found in \Cref{tbl:hpo_optimal}. The model training is executed using \textbf{AMD Ryzen 7 2700X, Eight-Core Processor $@$ 3.7 GHz processor with 16384 MB installed RAM}.\\


\begin{table}[ht]
    \footnotesize
    \centering
    % \resizebox {\textwidth}{!}
    {\begin{tabular}{ p{0.1\linewidth} p{0.2\linewidth} p{0.3\linewidth}}
    \hline
    Model & Training time [s] & Optimal Hyperparameter \\
    \hline
    DTR & 0.044 & None \\
    $\text{DTR}_{OPT}$ & 0.021  & {\tt min\_samples\_split = 7}\\
    &&{\tt min\_samples\_leaf = 10}\\
    &&{\tt max\_features = 12}\\
    &&{\tt max\_depth = 8}\\
    RFR & 4.112 & None \\
    $\text{RFR}_{OPT}$ & 3.431  & {\tt min\_samples\_split = 2}\\
    &&{\tt min\_samples\_leaf = 1}\\
    &&{\tt max\_features = 10}\\
    &&{\tt max\_depth = 120}\\
    &&{\tt n\_estimators = 100}\\
    ETR & 0.944 & None \\
    $\text{ETR}_{OPT}$ & 4.390  & {\tt min\_samples\_split = 9}\\
    &&{\tt min\_samples\_leaf = 1}\\
    &&{\tt max\_features = 12}\\
    &&{\tt max\_depth = 120}\\
    &&{\tt n\_estimators = 800}\\
    MLR & 0.004  & None\\
    \hline
    \end{tabular}}
\caption{Optimal hyperparameter with training time of each model}\label{tbl:hpo_optimal}
\end{table}

With the default hyperparameter, RFR takes the longest training time followed by ETR and DTR. This is expected as RFR uses greedy algorithm i.e. it looks for the best possible feature when splitting the node. ETR takes significantly shorter time to train as ETR randomly select for features when splitting the node. DTR takes the shortest training time as it only generates a single tree. However, in the case of optimised model, ETR takes a longer time to train compared to RFR. This is caused by the number of trees in the optimised model which is controlled by the parameter {\tt n\_estimator}, the optimised ETR model has 800 trees in comparison to 100 trees of RFR. It is also observed that the training time of optimised DTR model is halved as pruning the tree resulted in a simpler model to train. \\

To further investigate the effect of hyperparameter optimisation, the learning curve of each tree-based model is plotted. For DTR, generated model with default parameter will result in a model that heavily overfits the training data, which is evident from the large gap between the training error and validation error which indicated a high variance as shown in \Cref{fig:learn_curve_DTR_RMSE}. Regularisation i.e. parameter tuning of the DTR model helps balance between bias and variance by trading bias for variance. This is observed from the substantial reduction in the gap between the training and validation error from \Cref{fig:learn_curve_DTR_RMSE}. Additionally, the learning curve indicates that the most notable improvement in model performance occurs until around 1000 data points. Beyond this point, the enhancement in model performance becomes less substantial.\\

\begin{figure}[h]
    \centering
        \includegraphics[width=.95\textwidth]{02_figures/learning_curve_dtr.png}
        \caption{Learning curve of DTR}
        \label{fig:learn_curve_DTR_RMSE}
\end{figure}

The process of hyperparameter tuning for the Random Forest Regressor (RFR) model did not show any significant improvement in model performance. This outcome aligns with the findings of  \bcitet{Kuhn.2013} and \bcitet{Hastie.2009} which was discussed in \Cref{sec:rf_theo}. The most notable improvement on model performance is observed until around 750 points, after which the model appears to reach a plateau. Furthermore, there is noticeable variance in the RFR model, which indicates that the model will have a slight tendency to overfit.\\

\begin{figure}[h]
    \centering
        \includegraphics[width=.95\textwidth]{02_figures/learning_curve_rfr_rmse.png}
        \caption{Learning curve of RFR}
        \label{fig:learn_curve_RFR_RMSE}
\end{figure}

Hyperparameter tuning helps to reduce variance in the ETR model. But it does not have any major impact on model's performance. The ETR model reaches plateau beyond 1000 data points. Suggesting that adding more data points will not result in any significant increase in model performance.\\

\begin{figure}[h]
    \centering
        \includegraphics[width=.95\textwidth]{02_figures/learning_curve_etr_rmse.png}
        \caption{Learning curve of ETR}
        \label{fig:learn_curve_ETR_RMSE}
\end{figure}

In addition to the initial exploration in \Cref{sec:hpo}, it can be concluded that hyperparameter tuning for number of features and tree depth will have the biggest impact in affecting the model's performance. To improve training time, lower number of trees should be considered for RFR and ETR model.\\


\subsection*{Analysis of trained model}\label{sec:BBM_model_eval}

\subsubsection*{Feature Importance}

As discussed in \Cref{sec:rf_theo}, tree-based models are able to quantify the impact of each feature during the split process, this is performed using {\tt feature\_importances\_} feature in \scikit/ \bcitep{Kuhn.2013}. According to documentation by \bcitet{FabianPedregosa.2011}, it is computed as the mean and the standard deviation of accumulation of the impurity decrease within each tree i.e. total reduction of the criterion brought by a feature. Alternatively, it can be defined as how much a feature is used in each tree.\\

\begin{table}[ht]
    \scriptsize
    \centering
    \resizebox {\textwidth}{!}
    {\begin{tabular}{ p{0.15\linewidth} c| p{0.15\linewidth} c| p{0.15\linewidth} c}
    \hline
    $\text{DTR}_{OPT}$ & & $\text{RFR}_{OPT}$ & & $\text{ETR}_{OPT}$\\
    \hline
    Feature & Importance & Feature & Importance & Feature & Importance\\
    \hline
    {\tt heading} & 0.6563 & {\tt heading} & 0.4927 & {\tt cog} & 0.6410 \\
    {\tt cog} & 0.3183 & {\tt cog} & 0.4183 & {\tt heading} & 0.2707 \\
    {\tt draught} & 0.0105 & {\tt draught} & 0.0210 & {\tt truecurrentdir} & 0.0200\\
    {\tt truewinddir} & 0.0047 & {\tt curspeed} & 0.0104 & {\tt draught} & 0.0144 \\
    {\tt oceantemperature} & 0.0029 & {\tt waveperiod} & 0.0093& {\tt windwaveswellheight} & 0.0112\\
    {\tt surftemp} & 0.0025&{\tt truecurrentdir} & 0.0092 & {\tt curspeed} & 0.0110\\
    {\tt waveperiod} & 0.0019 & {\tt windwaveswellheight} & 0.0084& {\tt waveperiod} & 0.0095 \\
    {\tt truecurrentdir} & 0.0010 & {\tt surftemp} & 0.0075& {\tt windspeed} & 0.0053\\
    {\tt windwaveswellheight} & 0.0008 & {\tt truewinddir} & 0.0075& {\tt surftemp} & 0.0046 \\
    {\tt curspeed} & 0.0004 & {\tt truewavedir} & 0.0058& {\tt truewavedir} & 0.0045 \\
    {\tt windspeed} & 0.0004 & {\tt oceantemperature} & 0.0057& {\tt oceantemperature} & 0.0044\\
    {\tt truwavedir} & 0.0001 & {\tt windspeed} & 0.0056& {\tt truewinddir} & 0.0033\\
    \end{tabular}}
\caption{Feature importance of different models}\label{tbl:feature_importances}
\end{table}

The feature importances for all tree-based models shown in \Cref{tbl:feature_importances} indicated that the structure of the model is significantly influenced by the features {\tt heading} and {\tt cog}. This finding indicated that the models predicted the SOG by basis of ship movement direction i.e. heading and COG for a particular location. However, in a physical sense, it will be more insightful to consider the ship state and weather conditions that affect the prediction of the SOG.\\

\begin{figure}[h]
    \centering
        \includegraphics[width=.9\textwidth]{02_figures/dtr_ftr_importance_nodir.png}
        \caption{Feature importance of DTR}
        \label{fig:ftr_impo_dtr}
\end{figure}

Excluding ship heading and COG. The ship draught $T$ is regarded as the significant factors that affect the SOG prediction. This aligns with the theory of frictional resistance $R_F$ encountered by the ship, which is discussed in \Cref{sec:Calm_Resistance}. \Cref{eqn:R_f} is a function of wetted surface area of bare hull $S$. Deeper draught $T$ will result in more submerged area of the hull and this will consequently increase the frictional force $R_F$ of the ship. Given a constant supply of power to the ship propulsion system, the speed of the ship will decrease which is shown in \Cref{eqn:P_e}.\\

\begin{figure}[h]
    \centering
        \includegraphics[width=.9\textwidth]{02_figures/rfr_ftr_importance_nodir.png}
        \caption{Feature importance of RFR}
        \label{fig:ftr_impo_rfr}
\end{figure}

\begin{figure}[h]
    \centering
        \includegraphics[width=.9\textwidth]{02_figures/etr_ftr_importance_nodir.png}
        \caption{Feature importance of ETR}
        \label{fig:ftr_impo_etr}
\end{figure}

For weather states, both RFR and ETR considered current based information such as current speed and true current direction as the most significant weather factor that affect SOG prediction. This aligns to the proposed current correction methodology presented in \Cref{sec:SOG_corr} which states that the process of current correction to convert SOG to STW requires both the magnitude and direction of the current. The next influencing feature ranked by RFR and ETR model are wave related features which are significant wave height $H_{1/3}$, true wave direction, and the wave period {\tt waveperiod}. This corresponds to the added resistance due to wave $R_{AW}$ in the calculation of total resistance $R_{TOTAL}$ encountered by the ship. The wind related features, which are wind speed and its true direction, corresponds to added resistance due to wind force $R_{AA}$ and is found to be the least influential in SOG prediction in RFR and ETR model.\\

Based on the behaviour of the Random Forest Regressor (RFR) and Extra Trees Regressor (ETR) models, it can be inferred that waves have a more significant impact on the Speed Over Ground (SOG) compared to the influence of wind during the ship's journey. However, the Decision Tree Regressor (DTR) model demonstrates that temperature-related features, such as Sea Surface Temperature (SST) and air temperature above the ocean, have a more significant effect on SOG predictions than most other features. While the importance of temperature is not as pronounced as in RFR or ETR models, this finding suggests that the ship's SOG is implicitly influenced by the time of the travel or the season in which the journey takes place.\\

\subsubsection*{Structure of generated tree-based model}

\begin{figure}[h]
    \centering
        \includegraphics[width=.9\textwidth]{02_figures/dtr_mod_1tree.png}
        \caption{Structure of DTR}
        \label{fig:dtr_tree_hpov}
\end{figure}

To understand the effect of hyperparameter optimisation and feature importance, analysis on the structure of generated tree-based models will be performed. The shading in the nodes indicated the likelihood of the decision, where darker shading means greater likelihood. Each node indicates information on the splitting feature with the threshold, the SSR score, amount of samples and the predicted SOG value. Even after pruning, the tree can grow relatively large, therefore, the illustration of the trees is limited to a depth of {\tt max\_depth = 3} and for RFR and ETR, only the illustration of a specific tree in the forest will be shown.\\

The structure of the optimised decision tree shown in \Cref{fig:dtr_tree_hpov} show the effect of regularisation at the leaf node. For example, the leaf nodes that splits the feature, ocean temperature, does not completely minimise the SSR. This is caused by the hyperparameter tuning of the minimum samples at the leaf node, which was set at {\t min\_samples\_leaf = 10}, splitting these nodes further will cause subsequent leaf nodes that have less than 10 samples. In this figure, the significance of COG and ship heading can be clearly seen, as it is used to split many of the internal nodes in the tree.\\

\begin{figure}[h]
    \centering
        \includegraphics[width=.9\textwidth]{02_figures/rfr_mod_it1.png}
        \caption{Structure of RFR}
        \label{fig:rfr_tree1_hpov}
\end{figure}

The illustration of the first RFR tree is shown in \Cref{fig:rfr_tree1_hpov}. Similar to DTR, both COG and ship heading are regarded as the best features to split the internal node. In this tree of the forest, the effect of allowing full tree growth can be observed in the leaf node when splitting the feature current speed. This tree is able to minimise the SSR to its possible minimum value 0, and the leaf node cannot be further split as there are no more available samples. The effect of bagging for the dataset and feature selection in RFR can also be observed in this tree as the structure of this tree is completely different to DTR tree shown in \Cref{fig:dtr_tree_hpov}.\\

\begin{figure}[h]
    \centering
        \includegraphics[width=.9\textwidth]{02_figures/etr_mod_it1.png}
        \caption{Structure of ETR}
        \label{fig:etr_tree1_etr}
\end{figure}
The random selection of the feature to split of ETR can be seen in the structure of the first tree in an ETR in \Cref{fig:etr_tree1_etr}. Since both DTR and RFR uses greedy algorithm, i.e. it finds the possible splits which minimise the cost function, both model selected COG as the parent node. However, the randomness in feature selection of ETR can be clearly observed in this illustration, the model selects the true current direction as the parent node. Also, due to regularisation of ETR, the leaf node when splitting COG does not completely minimise the SSR. The split is not allowed since it does not meet the tuning criteria of {\tt min\_samples\_split = 9}.\\ 

\subsubsection*{Evaluation of k-fold cross-validation}

The performance of the model is evaluated using the training dataset using 10-fold cross validation. This means that the training will be repeated 10 times using 9 of the folds as training dataset, the remaining fold will be used as validation dataset. The results from k-folding validation process is shown in \Cref{fig:train_boxplot_r2_rmse}. The inside (orange) line represents the median i.e. $50\%$ of the score in k-folding. The top and the bottom of the box correspond to the first i.e. $25\%$ and third quartile i.e. $75\%$ respectively. The whiskers represent the lowest data point within the 1.5 Interquartile Range (IQR) of the lowest quartile and the highest point of data within 1.5 IQR of the upper quartile. The mean is indicated by the (green) triangle. Data points beyond the whisker range is shown as hollow circle.\\

\begin{figure}[ht]
    \centering
    \subfigure[k-fold $R^2$ validation performance]{\includegraphics[width=0.49\textwidth]{02_figures/kfold_r2_opt.png}} 
    \subfigure[k-fold RMSE validation performance]{\includegraphics[width=0.49\textwidth]{02_figures/kfold_rmse_opt.png}}
    \caption{Box plots of different models with default and optimised parameter in k-folding for training dataset}
    \label{fig:train_boxplot_r2_rmse}
\end{figure}

The box plots indicated that ETR achieved the best performance, the model is able to achieve $R^2$ score of around $91\%$ and RMSE of around 0.96 knots. The model is also relatively stable, which is indicated by the narrow box plots. RFR also achieved similar performance, achieving $R^2$ score of about $89\%$ and RMSE of approximately 1.00 knots and slightly worse stability. This behaviour may be caused due to the high variance shown from the learning curves shown in \Cref{fig:learn_curve_RFR_RMSE} and \Cref{fig:learn_curve_ETR_RMSE}. This means that the model will have slight tendency to overfit.\\

DTR greatly benefits from regularisation, the model achieves an increase of about $5\%$ for the $R^2$ score and a reduction from about 1.2 knots to 1.1 knots for the RMSE. To summarise, all tree-based models exhibited good fit performance with mean/median $R^2$ scores above $80\%$. However, the value of RMSE range is quite significant, the values lies between 1.00 to 1.20 knots across the models. To put this into scale, the mean SOG of the training data is at 16.91 knots as shown in \Cref{tbl:dataset_descriptive_pretraining}.\\  

\subsection*{Analysing the testing dataset}

Once the best model is determined, the model will be tested against the testing dataset. The testing dataset correspond to 957 datasets across 2021, the dataset for the whole year is indicated as $DS_{year}$. To investigate the effect of data points on the model performance, the dataset is split into two seasons, $DS_{summer}$, which corresponds to summer datasets and it represents data between May 2021 and October 2021, there are 454 data points between this period. Winter dataset, $DS_{winter}$, correspond to testing datasets between January 2021 and April 2021 as well as November 2021 and December 2021 which correspond to 503 data points. Any missing values which are present in the testing dataset will be using {\tt KNNImputer}.\\

\begin{table}
    \footnotesize
    \centering
    % \resizebox {\textwidth}{!}
    {\begin{tabular}{ p{0.21\linewidth} c c c c c c c c }
    \hline
    Features & Count & Mean & Std. & Min & 25\% & 50\% & 75\% & Max \\
    \hline
    \textbf{{\tt sog}} & 957 & 16.99 & 3.10 & 5.10 & 16.68 & 18.05 & 18.72 & 21.00\\
    \hline
    {\tt cog} & 957 & 196.73 & 86.72&	56.02 & 102.32& 185.22& 282.18& 319.85\\ 
    {\tt heading} & 957 & 187.88&	88.47&	67.90&	100.86&	124.65&	279.19&	319.85\\
    {\tt draught} & 957 & 5.23 & 0.19& 4.74& 5.11& 5.29& 5.38&5.66\\
    {\tt windspeed} & 957 & 6.45 & 3.04 & 0.40 & 4.11 & 6.13 &	8.21 & 15.85\\
    {\tt oceantemperature} & 957 & 282.27 & 6.48 & 267.25& 276.80& 281.91& 288.42& 295.70 \\
    {\tt waveperiod} & 957 & 3.40 & 0.88 & 1.67 & 3.06& 3.62& 4.22& 7.01\\
    {\tt surftemp} & 957 &283.22& 5.72& 273.15& 277.98& 282.73& 288.82 &294.92\\
    {\tt windwaveswellheight} &  957 & 0.77 & 0.54 & 0.08 &0.37 &	0.66 &	0.94 &  3.24  \\
    {\tt curspeed} & 957 &0.09 & 0.07& 0.00 & 0.05& 0.07 & 0.13 & 0.50\\
    {\tt truewinddir} & 957 & 91.39 & 56.23 &	0.03 & 38.80 &	95.25 & 142.83 & 179.86\\
    {\tt truecurrentdir} & 957 & 90.75 & 57.76 & 0.26 & 31.52 & 90.44 & 144.65 & 179.95 \\
    {\tt truewavedir} & 957 & 86.90 & 55.74& 0.06& 36.24 & 81.54 & 138.04 & 179.81 \\
    \hline
    \end{tabular}}
\caption{Descriptive statistics of $DS_{year}$}\label{tbl:testyear_dataset_descriptive}
\end{table}

\subsection*{Evaluation of testing datasets}

To evaluate model performance, the testing datasets $DS_{year}$, $DS_{summer}$ and $DS_{winter}$ will be passed through the optimised model with hyperparameter values presented in \Cref{tbl:hpo_optimal}. The model performance is summarised in \Cref{tbl:testing_dataset_sog_result}   

\begin{table}[ht]
    \footnotesize
    \centering
    % \resizebox {\textwidth}{!}
    {\begin{tabular}{ p{0.1\linewidth} p{0.1\linewidth} c c c c c c }
    \hline
    Model & Dataset & $R^2$ & expVar & MAE & RMSE & MAD & MAPE \\
    \hline
    & & [$\%$] & [$\%$] & [$kn$] & [$kn$] & [$kn$] & [$\%$]  \\ 
    $\text{DTR}_{OPT}$ & $DS_{year}$ & 90.10 & 90.12 & 0.629 & 0.975  & 0.420 & 4.21  \\
    & $DS_{winter}$ & 93.18 & 93.19 & 0.561 & 0.846 & 0.390 & 3.92 \\
    & $DS_{summer}$ & 85.69 & 84.90 & 0.704 & 1.100 & 0.451 & 4.52 \\
    $\text{RFR}_{OPT}$ & $DS_{year}$  & 96.59 & 96.60 & 0.335 & 0.572 & 0.187 & 2.29 \\
    & $DS_{winter}$ & \textbf{98.41} & \textbf{98.42} & \textbf{0.265} & \textbf{0.409} & \textbf{0.173} & \textbf{1.94} \\
    & $DS_{summer}$ & 94.02 & 94.14 & 0.412 & 0.710 & 0.215 & 2.68 \\
    $\text{ETR}_{OPT}$ & $DS_{year}$ & 96.82 & 96.82 & 0.347 & 0.553 & 0.214 & 2.35 \\
    & $DS_{winter}$ & 98.40 & 98.40 & 0.287 & 0.410 & 0.196 & 2.03 \\
    & $DS_{summer}$ & 95.49 & 94.68 & 0.413 & 0.676 & 0.239 & 2.70 \\
    MLR & $DS_{year}$ & 69.75 & 69.85 & 1.139 & 1.704 & 0.908 & 7.64 \\
    & $DS_{winter}$ & 68.16 & 68.17 & 1.129 & 1.828 & 0.871 & 7.94 \\
    & $DS_{summer}$ & 71.43 & 71.87 & 1.150 & 1.554 & 0.951 & 7.32 \\
    \hline
    \end{tabular}}
\caption{Performance indices for different modelling approach and different testing datasets}\label{tbl:testing_dataset_sog_result}
\end{table}

From \Cref{tbl:testing_dataset_sog_result}, it can be observed that all tree-based model is able to achieve good results on testing datasets. All tree-based model obtained $R^2$ scores above $90\%$ and is also able to obtain RMSE as low as 0.409 knots. In general, all tree based models performs better when $DS_{winter}$ datasets are used for testing. The results also show that ETR and RFR have nearly identical performance for SOG prediction. Both of the model achieved model fit score $R^2$ of about $96\%$ and RMSE of about $0.5-0.6$ knots. While the fit performance of DTR is comparable to both RFR and ETR, DTR makes more substantial errors during predictions, this can be seen from the MAE of DTR from figure \Cref{tbl:testing_dataset_sog_result}, DTR model makes larger error during the prediction, this pattern is consistent for all error metrics such as MAE, RMSE, MAD and MAPE, where DTR model exhibits errors that are twice as large as those of other models with exception to MLR.\\




\section{Introduction} \label{introduction}

The research on efficient vessel operation is a direction that is actively being pursued by marine industry stakeholder. Efficient vessel operation reduce  translates to increase in profitability of the operation. One of the determining factor will be the reduction of Fuel Oil Consumption (FOC). Fuel cost takes up considerable portion in ships operating cost. This is clearly indicated through findings made by Ronen \cite{Ronen.2011} and Stopford \cite{Stopford.2009}. The findings mentioned that FOC consumption of a large ship potentially constitute to 75\% of the total ship operating cost while the latter noted that FOC makes up to \(\frac{2}{3}\) of vessel voyage cost and over \(\frac{1}{4}\) of vessel's overall cost. \\

With that, maritime industry stakeholder actively search for inexpensive approach to reduce FOC. As such, they look into ways to optimise operational measure as technical solutions are expensive. The operational measures include the inclusion of weather/environmental routing, speed optimisation, trim optimisation and virtual (just-in/time) arrival policy \cite{Li.2022}. It is noted by {Be{\c{s}}ik{\c{c}}i} et al. \cite{BalBesikci.2016} that lowering ship speed will have the greatest impact in fuel economy, reducing the ship speed by $2-3$ Knots could halve the operating cost of shipping company \cite{Stopford.2009,Wijnolst.2009}. {Be{\c{s}}ik{\c{c}}i} et al. then further elaborated that the main cause of this is the nonlinear relationship between ship speed and fuel consumption. Ronen \cite{Ronen.1982} estimated that fuel consumption can be calculated through third order function of the ship speed. Wang et al. \cite{Wang.2012} verified this estimation.\\






%\begin{figure}[h]
%    \centering
%    \includegraphics[width=0.50\textwidth]{02_figures/logo_en.jpg}
%    \caption{A nice plot}
%    \label{fig:mesh1}
%\end{figure}


\chapter{Introduction} \label{chp:introduction}

Marine industry stakeholders are actively pursuing research on efficient ship operation. This research direction is motivated by the increasing price of fuel oil and stricter environmental regulations. The fuel aboard a ship is referred to as ``bunkers'' and accounts for a substantial portion of the vessel's operational expenses (OPEX). It is known that bunker fuel takes up more than 50\% of voyage costs and constitutes up to 75\% of the ship's total operating cost. It can be inferred that energy-efficient ship operations that could reduce fuel consumption translate to an increase in profitability \bcitep{Stopford.2009,Ronen.2011,Bialystocki.2016}. Furthermore, efficient operation also means the reduction of Greenhouse Gas Emissions (GHG). The most recent report by International Maritime Organisation indicated that GHG emissions from shipping make up 2.51\% of global emissions \bcitep{IMO.2020}. This mutual motivation aligns economic benefits with environmental compliance.\\

With that, maritime industry stakeholder actively searches for methods to ensure energy-efficient operation. Two approaches are considered, namely technical solutions and operational solutions. Technical solutions involve modification to the vessel's structure and power system. But these solutions are expensive, and it requires engineering innovations \bcitep{Yan.2021,Li.2022}. Because of this, stakeholders look for cheaper solutions to achieve energy-efficient operations. The answer for an inexpensive approach lies in the optimisation of operational measures, it carries less cost, and it does not require initial investments. Several recommended solutions for energy-efficient operation can be found in Ship Energy Efficiency Management Plan (SEEMP).\\ 

Significant emphasis is given in this study on the optimisation of ship speed as it is widely considered that ship speed has a substantial impact on fuel consumption. Different studies indicated that fuel consumption is correlated through a third-order, non-linear function of the ship speed \bcitep{Wang.2012,Ronen.2011,Psaraftis.2013,Du.2019}. The significant impact of ship speed on fuel consumption is further supplemented by reports and studies stating that reducing ship speed by about $2-3$ knots could halve the operating cost of shipping companies \bcitep{Stopford.2009,Wijnolst.2009}. Hence, energy-efficient operation is commonly achieved through the practice of slow steaming by shipping industry operators. \\

While inexpensive, optimising operational measures is not an easy and trivial task. Several factors ranging from vessel operational performance to varying weather conditions make it challenging to model the ship speed. Some fuel consumption models, which are based on historical data and ship parameters, lack generalisation capabilities, and it is sensitive towards noisy data. To address this problem, recent research turns towards data-driven approach i.e. machine learning (ML) approach to predict ship speed and fuel consumption. These studies reported success in their modelling, citing good generalisation capability and low prediction errors. Despite these successes, maritime experts find it difficult to accept models based on data-driven approach, as some data-driven models are complex as well as unintuitive and in some cases can violate basic physical knowledge of the vessel. The performance of the data-driven model is also greatly dependent on both data quantity and quality \bcitep{Yan.2021,Gkerekos.2019}.\\      

As such, prompted by volatility and ever-increasing bunker fuel price, developing a model that could accurately predict Fuel Oil consumption (FOC) could prove to be useful to maritime industry stakeholders. As stakeholders could make critical economical decisions at the most opportune moment without violating the stringent environmental regulations. \\

\section{Thesis objectives}\label{sec:objectives}

This thesis proposes an intuitive, data-driven modelling approach that considers varying ship states and environmental conditions to predict the fuel consumption of a vessel. To ensure the abundance of data during modelling, this thesis utilise data fused between Automatic Identification System (AIS) and weather data.\\

To achieve this, Grey Box Model (GBM) approach is selected. ML approach using tree-based regressor is considered to provide a certain degree of intuitiveness to predict speed over ground (SOG) over different journey periods using fused AIS and weather data. Predicted SOG is then converted to actual ship speed i.e. Speed Through Water (STW). STW will be used as the input for the modelling of Fuel Oil Consumption (FOC), which is carried out through Holtrop-Mennen estimation method \bcitep{Holtrop.1978,Holtrop.1982,Holtrop.1984}, a power estimation method based on hydrodynamic laws which consider resistance forces exerted by environmental conditions.\\

\pagebreak

The following Research Questions (RQs) could be raised during the development of the model :

\begin{itemize}
    \item \textbf{RQ1}: What are the steps that should be taken to optimise the predictive performance of the model?
    \item \textbf{RQ2}: Is it feasible to fuse AIS data and meteorological data to accurately predict the ship's SOG and subsequently FOC of the ship?
    \item \textbf{RQ3}: Which approximations and empirical equations are suitable to estimate the resistance forces required to estimate the power required by the ship? 
\end{itemize} 

\section{Thesis Boundaries}\label{sec:boundaries}

The following research boundaries are set throughout this thesis:

\begin{itemize}
    \item The weather information and AIS data are assumed to be true. Any uncertainties from AIS data and weather data are neglected. 
    \item The focus of this work is a detailed study of the performance and possible optimisation configuration of different tree-based predictors for SOG. As such, an exhaustive comparison study between different types of machine learning models will not be performed.
    \item In the case study, the approximation for incomplete ship parameters and dimensions is based on a similar type of ship with nearly identical dimensions. 
\end{itemize}

\section{Thesis Contributions}\label{sec:contributions}

The GBM approach using the fusion of AIS data and weather data provides the following contributions : 

\begin{itemize}
    \item Economical and independent data source.
    \item Robust modelling approach that requires minimal data pre-processing and minimal model configuration.
    \item Comprehensible model that adheres to physical principles and hydrodynamic laws of the vessel.  
\end{itemize} 

\pagebreak

\section{Thesis Structure}\label{sec:structure_thesis}

The thesis is organised with the following structure:\\

\textbf{\Cref{chp:introduction}} introduces the problem statement and described the objective and boundaries of the thesis. The novelty of this thesis is declared in this chapter.\\

\textbf{\Cref{chp:theory}} explains the fundamental aspects of the methodologies used to develop the Black Box Model (BBM) and the White Box Model (WBM). \Cref{sec:litreview} includes literature review of relevant past and present research. The fundamentals of the tree-based model will be discussed in \Cref{sec:tree_intro}, basic explanation of the parameters used in AIS and weather data will be given in \Cref{sec:ais_theo} and \Cref{sec:weather_theo}. \Cref{sec:holtrop_mennen_calc} presents the empirical formulas and parameters used to estimate fuel consumption used by the ship based on various literature studies.\\ 

\textbf{\Cref{chp:method}} discusses the methodology used to develop tree-based model used for SOG prediction. The discussion comprises analysis of training data, feature selection and reduction and selection of tuning parameters of the model. The methodology to estimate resistance for ship power estimation will be discussed in this chapter as well.\\ 

\textbf{\Cref{chp:result_and_discussion}}, the GBM model will be evaluated using appropriate performance metrics and their effectiveness will be discussed. The review of the strength and limitations concerning the GBM method will be discussed here.\\

\textbf{\Cref{chp:outlook}} The summary of this study and reflections on the research process will be presented here. 




 





%\begin{figure}[h]
%    \centering
%    \includegraphics[width=0.50\textwidth]{02_figures/logo_en.jpg}
%    \caption{A nice plot}
%    \label{fig:mesh1}
%\end{figure}


\chapter{Introduction} \label{chp:introduction}

The research on efficient ship operation is a direction that is being actively pursued by marine industry stakeholders. This research direction is motivated by increasing price of fuel oil and stricter environmental regulations. Fuel onboard ship is referred to as ``bunkers'' and it takes up considerable portion of ship's Operational Expenses (OPEX). Taking up more than 50\% of voyage cost and constituting up to 75\% of the ship's total operating cost. These significant figures indicate that energy efficient ship operation that could reduce fuel consumption translate to increase in profitability \citep{Stopford.2009,Ronen.2011,Bialystocki.2016}. Additionally, efficient operation also means reduction of Green House Gas (GHG) emissions. Most recent report by International Maritime Organisation indicated that GHG emission from shipping makes up 2.51\% of global emission \citep{IMO.2020}. This alignment in motivation implies that through energy efficient ship operation, marine industry stakeholders gain economic benefits while adhering to stringent environmental regulations. \\

With that, maritime industry stakeholder actively searches for methods to ensure energy efficient operation. Two approaches are considered, namely technical solutions and operational solutions. Technical solutions involve modification to vessel's structure and power system. Technical solution is however costly, and it requires engineering innovations \citep{Yan.2021,Li.2022} and because of this, stakeholders looks for cheaper solutions to achieve energy efficient operation. The answer for an inexpensive approach lies in optimisation of operational measures, it carries less cost and do not require initial investments. Several recommended solutions can be found in Ship Energy Efficiency Management Plan (SEEMP).\\ 

However, greater focus will be given in this thesis towards optimising ship speed as reduction of ship speed have the greatest impact on fuel consumption. Different studies indicated that fuel consumption is correlated through a third-order, nonlinear function of the ship speed \citep{Wang.2012,Ronen.2011,Du.2019}. The significant impact of ship speed on fuel consumption is further supplemented by reports which state that reducing ship speed by about $2-3$ knots could halve the operating cost of shipping companies \citep{Stopford.2009,Wijnolst.2009}. For these reason, slow steaming is the measure that is most widely adopted by shipping operator. \\

While inexpensive, optimising operational measures is not an easy and trivial task. Several factors ranging from vessel operational performance to varying weather conditions makes it challenging to model the ship speed. Some fuel consumption model, that are based on historical data and ship parameters, lacks generalisation capabilities and it is sensitive towards noisy data. To address this problem, recent research turns towards data-driven approach i.e. machine learning approach to predict ship speed and fuel consumption. These studies reported success in their modelling, citing good generalisation capability and low prediction errors. In spite of these successes, maritime experts find it difficult to accept model based on data driven approach, as some data-driven models are complex as well as unintuitive and in some cases can violate basic physical knowledge of the vessel. The performance of the data-driven model is also greatly dependent in both data quantity and quality \citep{Yan.2021,Gkerekos.2019}.\\      

As such, prompted by volatility and ever-increasing bunker fuel price, developing a model that could accurately predict Fuel Oil consumption (FOC) could prove to be useful to maritime industry stakeholders. As stakeholders could make critical economical decisions at the most opportune moment without violating the stringent environmental regulations. \\

\section{Thesis Objective}\label{sec:objectives}

This thesis proposes an intuitive, data-driven modelling approach of that considers varying ship state and environment conditions to predict fuel consumption. To ensure abundance of data during modelling, this thesis utilise data fused between Automatic Identification System (AIS) and weather data.\\

To achieve this, Grey Box Model (GBM) approach is selected. Machine learning approach using random forest regressor (RFR) is considered to provide certain degree of intuitiveness to predict ship over ground (SOG) over different journey period using fused AIS and weather data. Predicted SOG is then converted to actual ship speed i.e. Speed Through Water (STW). STW will be used as the input for modelling of Fuel Oil Consumption (FOC), which is carried out through Holtrop-Mennen estimation method \citep{Holtrop.1978,Holtrop.1982,Holtrop.1984}, a power estimation method based on hydrodynamic laws which considers resistance forces exerted by environmental conditions.\\

Following Research Questions (RQs) could be raised during the development of the model :

\begin{itemize}
    \item \textbf{RQ1}: What are the steps that should be taken to optimise the predictive performance of the model ?
    \item \textbf{RQ2}: Is it feasible to fuse AIS data and meteorological data to accurately predict the ship's SOG and subsequently FOC of the ship ?
    \item \textbf{RQ3}: Which approximations and empirical equations are suitable to estimate the resistance forces required to estimate power required by the ship ? 
\end{itemize} 

\section{Thesis Boundaries}\label{sec:boundaries}

The following research boundaries are set throughout this thesis:

\begin{itemize}
    \item Due to the continuous nature of the SOG, only the regression aspect of Random Forest (RF) will be considered.
    \item The focus of this work is a detailed study on the performance and possible optimisation configuration of different tree based predictor for SOG. As such, exhaustive comparison study between different type machine learning models will not be performed.
    \item In the case study, the approximation for ship parameter and dimension is based on similar type of ship with nearly identical dimensions. 
\end{itemize}

\section{Thesis Contributions}\label{sec:contributions}

The GBM approach using fusion of AIS data and weather data provide the following contributions : 

\begin{itemize}
    \item Economical and independent data source.
    \item Robust modelling approach that requires minimal data pre-processing and minimal model configuration.
    \item Comprehensible model that adhere to physical principles and hydrodynamic laws of the vessel.  
\end{itemize} 

\section{Thesis Structure}\label{sec:structure_thesis}

The thesis is organised with the following structure:\\

\textbf{\Cref{chp:introduction}} introduces the problem statement and described the objective and boundaries of the thesis. The novelty to this thesis is declared in this section.\\

\textbf{\Cref{chp:theory}} The fundamental aspects of the methodologies used to develop the model will be explained in this chapter.\Cref{sec:litreview} include literature review pertaining to relevant past and present research. The fundamentals to tree based model will be discussed in \Cref{sec:tree_intro}, basic explanation on the parameters used in AIS and weather data will be given in \Cref{sec:ais_theo} and \Cref{sec:weather_theo}. \Cref{sec:foc_calc} presents the empirical formulas and parameters used to estimate fuel consumption used by the ship based on various literature studies.\\ 

\textbf{\Cref{chp:method}} discuss the methodology used to develop tree-based model used for SOG prediction. The discussion comprises analysis of training data, feature selection and reduction and selection of tuning parameter of the model.\\ 

\textbf{\Cref{chp:result}}, the GBM model will be evaluated using appropriate performance metrics and their effectiveness will be discussed. The review of the strength and limitation concerning the GBM method will be discussed here.\\

\textbf{\Cref{chp:outlook}} The summary of this study and reflections of the research process will be presented here. 




 





%\begin{figure}[h]
%    \centering
%    \includegraphics[width=0.50\textwidth]{02_figures/logo_en.jpg}
%    \caption{A nice plot}
%    \label{fig:mesh1}
%\end{figure}


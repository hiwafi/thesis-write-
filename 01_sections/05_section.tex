\chapter{Summary and Outlook} \label{chp:outlook}

\section{Conclusion}\label{sec:conclusion}

This thesis introduces a comprehensive approach that combines data-driven techniques with empirical models to estimate FOC for a sailing vessel. The optimised machine learning model effectively forecasts FOC for vessels navigating at varying speeds, draughts, and weather conditions. The outcomes substantiate the viability of integrating AIS data and weather data for SOG prediction, which will then be used for FOC estimation. Technical details about the ship can be derived from AIS data. Along with suitable approximatiions from suitable and relenvant literature, FOC can be forecasted using the empirical formulas proposed by Holtrop-Mennen. The results of predicted FOC can be used to generate bunker-to-fuel functions to estimate FOC for varying STW. The main findings and conclusion are presented in the following parts of this chapter.\\

\subsection*{\emph{Endogeneity and Multicollinearity in BBM}}



\subsection*{\emph{Importance of data}}


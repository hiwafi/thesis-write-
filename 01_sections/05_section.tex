\chapter{Summary and Outlook} \label{chp:outlook}

\section{Conclusion}\label{sec:conclusion}

This thesis introduces a comprehensive approach that combines data-driven techniques with empirical models to estimate FOC for a sailing vessel. The optimised machine learning model effectively forecasts FOC for vessels navigating at varying speeds, draughts, and weather conditions. The outcomes substantiate the viability of integrating AIS data and weather data for SOG prediction, which will then be used for FOC estimation. Technical details about the ship can be derived from AIS data. Along with suitable approximations from suitable and relevant literature, FOC can be forecasted using the empirical formulas proposed by Holtrop-Mennen. The results of predicted FOC can be used to generate bunker-to-fuel functions to estimate FOC for varying STW. The main findings and conclusion are presented in the following parts of this chapter.\\

\subsection*{\emph{Necessity of feature correlation in BBM}}

Machine learning-based FOC models rely heavily on feature engineering such as feature selection and feature importance identification. A prominent approach is the high correlation filter analysis, which involves the identification and elimination of highly correlated features. However, applying this filter necessitates careful consideration. Removing a feature should primarily be based on the understanding of physical and vessel-related knowledge.\\

Even though tree-based model inherently solves the problem of correlations between features and resist collinearity \bcitep{Yan.2020}, feature selection might still be necessary to simplify the generated model and potentially enhance its performance. A more complex model with more features does not necessarily entail better performance and could be detrimental to computational cost. Additionally, it could be susceptible to endogeneity, defined as a correlation between the independent or observed variables in the model and the unobserved additive error term \bcitep{Danaf.2023}. Feature importance identification, which is an inherent benefit of tree-based model, serves as a valuable approach not only for conducting post-training feature selection but also for verifying the model's alignment with the domain of physical and vessel-related knowledge thereafter.\\ 

\subsection*{\emph{Impact of data quantity, quality and resolution}}

As discussed in \Cref{sec:key_findings}, the quality and volume of data are correlated factors that greatly affect model performance. This should take precedence over hyperparameter optimisation. To put this into perspective, consider the instance of the STW distribution within the case study shown in \Cref{fig:hist_resistance_power_yr}. Adding more data points in the higher speed range would not necessarily enhance the model's ability to predict FOC at lower speeds—an issue already evident in the case study. Addressing the challenge posed by AIS data, both in terms of its volume and quality, could potentially be tackled by exploring alternative data sources such as S-AIS. Unlike T-AIS, S-AIS circumvents the inherent range limitations and is particularly advantageous for scenarios involving ocean-going vessels.\\   

Also, the hourly temporal resolution of AIS data, as opposed to noon data, presents a distinct advantage when predicting cases within narrower periods. The model trained in hourly resolution in this thesis proved to be effective in forecasting FOC within seasonal or yearly intervals. The influence of temporal resolution is also shown in the work of \bcitet{Gkerekos.2019}. For an equivalent volume of data, the noon data represents approximately 2.5 years' worth of information, whereas the sensor-based data, characterised by hourly resolution, corresponds to three months of information. The resulting assessments indicate that the model generated using sensor-based data exhibits fewer errors during predictions. For instance, considering the ETR model, the MAE for sensor-based data is calculated as 0.534 T/day, whereas for the noon data, the MAE is observed to be 1.434 T/day.\\

\subsection*{\emph{Power estimation method using Holtrop-Mennen method}}


The use Holtrop-Mennen method as power estimation method in this thesis has proven to be effective in estimating the energy required for operation. Missing input values that are not available can be approximated using formulas from different literature or estimated from similar case studies. However, the approximations are possible sources of errors and deviations for the estimation. If results from a towing tank resistance test are available, performing interpolation on the measurement values rather would be the preferable approach \bcitep{XiaoLang.2020}. Due nonlinearity of the power estimation method, it would be necessary to ensure the best possible accuracy and precision during the modelling of SOG or STW, especially if the vessel is sailing at high speed e.g. in the context of a merchant ship, the vessel sails at around 19 to 20 knots.\\ 

The minimisation of error terms for power estimation is crucial for scenarios such as Short Sea Shipping (SSS) which is defined as the maritime transport of goods and passengers by sea over enclosed seas \bcitep{vandenBos.2018}. Consider the following scenarios: an intra-regional journey of a feeder vessel that consumes 59.3 T of bunker across different legs in her journey \bcitep{Schyen.2015} and an ocean-going 8000 TEU vessel travelling from Yantian (YT) to Los Angeles (LA) which could consume up to 147 T/d of bunker \bcitep{Wang.2012}. The effect of any error terms will be more significant for the SSS scenario. For identical total sailing distances, the prediction error from each journey in SSS accumulates until it reaches the same distance covered by an ocean-going vessel.\\

\subsection*{\emph{Strength and limitation of GBM approach for prediction of energy-efficient operation}}

The use of the Random Forest Regressor model as predictor for SOG of the Black Box Model has proven to be effective, slight performance improvement can be extracted when using Extra Trees Regressor. The approach requires minimal data pre-processing and minimal model configuration and the low variance in performance across different datasets showcased its robustness. The feature importance identification feature available to tree-based models provides implicit feature selection as well as an analysis tool to check whether the model obeys the physical domain knowledge of the vessel.\\

The White Box Model which incorporates the power estimation method by Holtrop-Mennen method further ensures that the energy estimation adheres to the physical principles and hydrodynamic laws of the vessel. The power estimation method can be used to plot bunker-to-speed functions to estimate the energy required for different operating speeds.\\

The combination with WBM diminishes the advantage of a Black Box Model which does not require any additional domain knowledge of the vessel. Additionally, the sequential approach will result in prediction errors that will be carried over during energy estimation.\\ 

\pagebreak

\section{Research outlook}\label{sec:research_outlook}

Thus far, all the research questions outlined in the introduction have been addressed. Within the established research boundaries, measures have been undertaken to enhance model performance. The demonstrated efficiency of the model in predicting SOG and FOC demonstrated the viability of fusion between AIS and weather data. However, there remain prospects to further refine the proposed methodology.\\

\subsection*{\emph{Improvement to BBM}}

In this thesis, the bagging ensemble tree-based model is used for the BBM. However, there are other types of tree-based model which uses the boosting ensemble strategy, which trains decision trees in sequence and improves the performance of trees step by step using the information of fitting errors and negative gradient. Research by \bcitet{Li.2022} demonstrates encouraging outcomes, suggesting that adopting this tree growth strategy could potentially enhance the model's performance.\\


\subsection*{\emph{Improvement of WBM}}

To the best of the author's knowledge, there is no existing research that offers a systematic conversion approach from SOG to STW. This conversion holds particular significance within this model, given that STW is a fundamental component for energy estimation. The methodology adopted in this thesis solely accounts for current as a factor for the SOG to STW conversion. However, in actuality, this conversion could be influenced by additional factors such as wind and wave effects, water depth, and potential hull fouling. Therefore, this is a potential research gap that may be pursued to further improve the energy estimation during vessel operation.\\ 

To further improve the accuracy of energy prediction, interpolation from measurements of towing resistance test \bcitet{XiaoLang.2020}, or possibly calculated resistance from CFD simulations should be performed. While this may decrease the usage generalisability of the proposed methodology, the specificity of this method could potentially improve the accuracy of energy prediction and enable a closer simulation of real-world sailing conditions.\\
















